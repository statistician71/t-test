\documentclass{article}
\usepackage{color}
\usepackage{Sweave}
\begin{document}
\Sconcordance{concordance:github_t-test.tex:github_t-test.Rnw:%
1 2 1 1 0 22 1 1 4 13 0 1 2 3 1 1 4 13 0 1 2 5 1 1 4 13 0 1 2 3 1 1 4 %
13 0 1 2 14 1}


\begin{titlepage}
    \begin{center}
        \vspace*{1cm}
        
        \LARGE
       \textbf{\textcolor{magenta}{Compare the Control and Treatment groups using two sample t-test}}
        
        
      
        
        
        
     \end{center}
\end{titlepage}





\section*{\textcolor{blue}{Compare the Control and Treament group for (General):}}
\begin{Schunk}
\begin{Soutput}
	Two Sample t-test

data:  Control and Treat
t = 0.1888, df = 4, p-value = 0.5703
alternative hypothesis: true difference in means is less than 0
95 percent confidence interval:
    -Inf 1253.62
sample estimates:
mean of x mean of y 
     1768      1666 
\end{Soutput}
\end{Schunk}
\textbf{Comments:} So, according to the p-value (\textcolor{red}{0.5703}) it can be said that the Control and Treatment group have no significant difference.


\section*{\textcolor{blue}{Compare the Control and Treament group for (Mob):}}
\begin{Schunk}
\begin{Soutput}
	Two Sample t-test

data:  Control and Treat
t = -0.0569, df = 4, p-value = 0.4787
alternative hypothesis: true difference in means is less than 0
95 percent confidence interval:
     -Inf 60.79479
sample estimates:
mean of x mean of y 
 66.00000  67.66667 
\end{Soutput}
\end{Schunk}
\textbf{Comments:} So, according to the p-value (\textcolor{red}{0.4787}) it can be said that the Control and Treatment group have no significant difference.




\section*{\textcolor{blue}{Compare the Control and Treament group for (WDes):}}
\begin{Schunk}
\begin{Soutput}
	Two Sample t-test

data:  Control and Treat
t = -0.1447, df = 4, p-value = 0.446
alternative hypothesis: true difference in means is less than 0
95 percent confidence interval:
     -Inf 100.7293
sample estimates:
mean of x mean of y 
 113.6667  121.0000 
\end{Soutput}
\end{Schunk}
\textbf{Comments:} So, according to the p-value (\textcolor{red}{0.446}) it can be said that the Control and Treatment group have no significant difference.


\section*{\textcolor{blue}{Compare the Control and Treament group for (DE):}}
\begin{Schunk}
\begin{Soutput}
	Two Sample t-test

data:  Control and Treat
t = 0.2239, df = 4, p-value = 0.5831
alternative hypothesis: true difference in means is less than 0
95 percent confidence interval:
     -Inf 547.1329
sample estimates:
mean of x mean of y 
 254.3333  202.3333 
\end{Soutput}
\end{Schunk}
\textbf{Comments:} So, according to the p-value (\textcolor{red}{0.5831}) it can be said that the Control and Treatment group have no significant difference.



\bigskip




\textbf{\textcolor{red}{Notes:}}\\

Here, we have total three category for each of the situation (e.g. General, Mob, WDes and DE). The categories are "Low", "Medium" and "High". I think we should use the t-test for each of the category for each of the situation separately. To do this we need at least two observation for each of the categories.
 

\end{document}
